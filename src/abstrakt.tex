\vfill
\begin{center}
{\it \large Abstrakt}
\vspace{0.2cm}

\begin{minipage}{0.8\textwidth}{
Tato práce se zabývá sledováním vozidel pomocí více nepřekrývajících se kamer. Cílem je návrh a realizace systému, který je schopný rozpoznat vozidla v městském prostředí a sledovat jejich polohu. Kamery mají objektiv typu rybí oko a jsou umístěny v pouličních lampách. Představujeme hlubokou neuronovou síť pro detekci vozidel využívající informace z videa, algoritmus pro sledování vozidel na jedné kameře založený na optical flow, hlubokou neuronovou síť pro počítání podobností mezi vozidly a pravděpodobnostní grafovou reprezentaci města. Provedené experimenty reálného světa ověřily schopnosti celého systému.
\\
\\
\textbf{Klíčová slova}: Detekce objektů, sledování přes více kamer, objektiv rybí oko
}
\end{minipage}
\end{center}
\vfill
\vspace{1cm}
\newpage{}